\documentclass[%
	11pt,
	a4paper,
	utf8,
	%twocolumn
]{article}	

\usepackage{style_packages/podvoyskiy_article_extended}


\begin{document}
\title{Конспект по книге Гудфеллоу <<Глубокое обучение>>\footnote{Гудфеллоу Я., Бенджио И., Курвилль А. Глубокое обучение. -- М.: ДМК Пресс, 2018. -- 652 с. }}

\author{}

\date{}
\maketitle

\thispagestyle{fancy}

\tableofcontents

\section{Численные методы}

\section{Основы машинного обучения}

\subsection{Точечная оценка}

Точечное оценивание -- это попытка найти единственное <<наилучшее>> предсказание интересующей величины. Пусть $ \{ x^{(1)}, \ldots, x^{(m)} \} $ -- множество $ m $ независимых и одинаково распределенных точек. \emph{Точечной оценкой}, или \emph{статистикой}, называется любая функция этих данных
\begin{align*}
	\theta_m = g(x^{(1)}, \ldots, x^{(m)}).
\end{align*}

В этом определении не требуется, чтобы $ g $ возвращала значение, близкое к истинному значению $ \theta $, ни даже чтобы область значений $ g $ совпадала со множеством допустимых значений~$ \theta $.

Алгоритм $ k $-групповой перекрестной проверки применяется для оценивания ошибки обобщения алгоритма обучения $ A $, когда имеющийся набор данных $ \mathbb{D} $ \emph{слишком мал} для того, чтобы простое разделение на обучающий и тестовый или обучающий и контрольный наборы могло дать точную оченку ошибки обобщения, поскольку среднее значение потери $ L $ на малом тестовом наборе может иметь высокую дисперсию. 

\subsection{Смещение}

\emph{Смещение оценки} определяется следующим образом
\begin{align*}
	\text{bias}(\hat{\theta}_m) = \mathbb{E}( \hat{\theta}_m ) - \theta, 
\end{align*}
где математической ожидание вычисляется по данным (рассматривается как выборка из случайной величины), а $ \theta $ -- истинное значение параметра, которое определяет порождающее распределение.

Оценка $ \hat{\theta} $ называется \emph{несмещенной}, если 
\begin{align*}
	\text(\hat{\theta}_m) = 0, \ \text{т.е.} \ \mathbb{E}(\hat{\theta}_m) = \theta.
\end{align*}

Оценка $ \hat{\theta}_m $ называется \emph{асимптотически несмещенной}, если
\begin{align*}
	\lim_{m \to \infty} \text{bias}(\hat{\theta}_m) = 0, \ \text{т.е.} \lim_{m \to \infty} \mathbb{E}(\hat{\theta}_m) = \theta.
\end{align*}


\subsection{Дисперсия}

Для определения смещения мы вычисляли математичесвкое ожидание оценки, но точно так же можем вычислить и ее дисперсию. \emph{Дисперсией оценки} называется выражение
\begin{align*}
	\text{Var}(\hat{\theta}).
\end{align*}

\emph{Стандартной ошибкой} $ \text{SE}(\hat{\theta}) $ называется квадратный корень из дисперсии.

Воспользовавшись центральной предельной теоремой, согласно которой среднее имеет приблизительно нормальное распределение, можем применить стандартную ошибку для вычисления вероятности того, что истинное математическое ожидание находится в выбранном интервале. Например, \emph{95-процентный доверительный интервал} вокруг выборочного среднего (вокру оценки) $ \hat{\mu}_m = \dfrac{1}{m} \sum\limits_{k=1}^{n} x^{(i)} $ определяется формулой
\begin{align*}
	(\hat{\mu}_m - 1.96 \, \text{SE} (\hat{\mu}_m), \hat{\mu}_m + 1.96 \, \text{SE} (\hat{\mu}_m))
\end{align*}
при нормальном распределении со средним $ \hat{\mu}_m $ и дисперсией $ \text{SE}(\hat{\mu}_m)^2 $.

NB: В экспериментах по машинному обучению принято говорить, что алгоритм $ A $ лучше алгоритма $ B $, если верхняя граница 95-процентного доверительного интервала для ошибки алгоритма $ A $ меньше нижней границы 95-процентного доверительного интервала для ошибки алгоритма $ B $.

\subsection{Поиск компромисса между смещением и дисперсией для минимизации среднеквадратической ошибки}

Что, если имеются две оценки, у одной из которых больше смещение, а у другой дисперсия? Какую выбрать? 

Самый распространенный подход к выбору компромиссного решения -- воспользоваться \emph{перекрестной проверкой}. Эмпирически продемонстрировано, что перекрестная проверка дает отличные результаты во многих реальных задачах.

Можно также сравнить среднекваратическую ошибку (MSE) обеих оценок
\begin{align*}
	\text{MSE} = \mathbb{E} [ ( \hat{\theta}_m - \theta )^2] = \text{bias}(\hat{\theta}_m)^2 + \text{Var}(\hat{\theta}_m)
\end{align*}

Желательной является оценка с малой MSE, именно такие оценки держат под контролем и смещение, и дисперсию. Соотношение между смещением и дисперсией тесно связано с возникающими в машинном обучении понятиями емкости модели, недообучения и переобучения.

Если ошибка обобщения измеряется посредством MSE (и тогда смещение и дисперсия становятся важными компонентами ошибки обобщения), то увеличение емкости (то есть \emph{усложение модели}) влечет за собой \emph{повышение дисперсии} и \emph{снижение смещения}.


\subsection{Состоятельность}

Обычно нас интересует также поведение оценки по мере роста размера обучающего набора. В частности, мы хотим, чтобы при увеличении числа примеров точечные оценки сходились к истинным значениям соответствующих параметров.

Формально это записывается в виде (условие состоятельности)
\begin{align*}
	\hat{\theta}_m \stackrel{\mathbf{P}}{\longrightarrow} \theta, \ (m \to \infty)
\end{align*}

Иногда это условие называют \emph{слабой состоятельностью}, понимая под \emph{сильной состоятельностью} сходимость \emph{почти наверное} $ \hat{\theta} $ к $ \theta $.

{\color{blue}Состоятельность гарантирует, что смещение оценки уменьшается с ростом числа примеров}. Однако обратное неверно -- {\color{red}из асимптотической несмещенности не вытекает состоятельность}. Рассмотрим, к примеру, оценивание среднего $ \mu $ нормального распределения $ N(x; \mu, \sigma^2) $ по набору данных, содержащему $ m $ примеров: $ \{ x^{(1)}, \ldots, x^{(m)} \} $.

Можно было бы взять в качестве оценки первый пример: $ \hat{\theta} = x^{(i)} $. В таком случае $ \mathbb{E}(\hat{\theta})_m = \theta $, поэтому оценка является несмещенной вне зависимости от того, сколько примеров мы видели. Отсюда, конечно, следует, что оценка асимптотически несмещенная. Но она не является состояительной, т.к. \emph{неверно}, что $ \hat{\theta}_m \to \theta, \ (m \to \infty) $.

\subsection{Оценка максимального праводподобия}

Рассмотрим множества $ m $ примеров $ \mathbb{X} = \{ x^{(1)}, \ldots, x^{(m)} \} $, независимо выбираемых из неизвестного порождающего распределения $ p_{data}(x) $.

Обозначим $ p_{model}(x; \theta) $ параметрическое семейство распределений вероятности над одним и тем же пространством, индексированное параметром $ \theta $. 

Тогда оценка максимального правдоподобия для $ \theta $ определяется формулой
\begin{align*}
	\theta_{ML} = \argmax_{\theta} \, p_{model} (\mathbb{X}; \theta) = \argmax_{\theta} \, \prod_{i=1}^{m} p_{model}(x^{(i)}; \theta)
\end{align*}

Такое произведение большого числа вероятностей по ряду причин может быть неудобно. Например, оно подвержено \emph{потере значимости}. Для получения эквивалентной, но более удобной задачи оптимизации заметим, что взятие логарифма правдоподобия не изменяет $ \argmax $, но преобразует произведение в сумму
\begin{align*}
	\theta_{ML} = \argmax_{\theta} \, \sum_{i=1}^{m} \log p_{model}(x^{(i)}; \theta)
\end{align*}

Поскольку $ \argmax $ не изменяется при умножении функции стоимости на константу, мы можем разделить правую часть на $ m $ и получить выражение в виде математического ожидания относительно эмпирического распределения $ \hat{p}_{data} $, определяемого обучающими данными
\begin{align*}
	\theta_{ML} = \argmax_{\theta} \, \mathbb{E}_{x \, \sim \, \hat{p}_{data}} [\, \log p_{model} (x; \theta) \,]
\end{align*}

{\color{blue}Один из способов интерпретации оценки максимального правдоподобия состоит в том, чтобы рассматривать ее как минимизацию дивергенции (расхождения) Кульбака-Лейблера между этими эмпирическим распределением $ \hat{p}_{data} $, определяемым обучающим набором, и модельным распределением.}

Дивергенция Кульбака-Лейблера определяется формулой
\begin{align*}
	D_{KL}(\hat{p}_{data} \, || \, p_{model}) = \mathbb{E}_{ x \, \sim \, \hat{p}_{data} } [\, \log \hat{p}_{data}(x) - \log p_{model}(x) \, ]
\end{align*}

Первый член разности в квадратных скобках зависит только от порождающего данные процесса, но не от модели. Следовательно, при обучении модели, минимизирующей дивергенцию КЛ, мы должны минимизировать только величину
\begin{align*}
	- \mathbb{E}_{ x \, \sim \, \hat{p}_{data} }[\, \log p_{model}(x) \,],
\end{align*}
а это, конечно, то же самое, что максимизация величины $ \theta_{ML} = \argmax_{\theta} \, \mathbb{E}_{x \, \sim \, \hat{p}_{data}} [\, \log p_{model} (x; \theta) \,] $.

NB: То есть, другими словами задача максимизации правдоподобия эквивалентна задаче минимизации дивергенции Кульбака-Лейблера между эмпирическим распределением $ \hat{p}_{data} $ и модельным распределением $ p_{model} $.





% Источники в "Газовой промышленности" нумеруются по мере упоминания 
\begin{thebibliography}{99}\addcontentsline{toc}{section}{Список литературы}
	\bibitem{ramalho:python-2022}{\emph{Рамальо Л.} Python -- к вершинам мастерства: Лаконичное и эффективное программирование. -- М.: МК Пресс, 2022. -- 898 с.}
	
	\bibitem{heydt:pandas-2019}{\emph{Хейдт М., Груздев А.} Изучаем pandas. -- М.: ДМК Пресс, 2019. -- 682 с.}
\end{thebibliography}

%\listoffigures\addcontentsline{toc}{section}{Список иллюстраций}

%\lstlistoflistings\addcontentsline{toc}{section}{Список листингов}

\end{document}
